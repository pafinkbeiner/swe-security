%!TEX root = ../dokumentation.tex

\chapter{Einleitung}

\section{Einführung in verteilte Systeme}
Für den Begriff \glqq verteilte Systeme \grqq{} liegt keine eindeutige Definition vor. Verschiedene Autoren definieren den Begriff der verteilten Systeme leicht unterschiedlich.
Nach A. Tanenbaum aus dem Jahre 2003 ist ein verteiltes System eine Menge voneinander unabhängiger Computer, die dem Benutzer wie ein einzelnes kohärentes System
erscheinen \cite{Mandl.2009}. Jeder Baustein kann seine Instanz auf unterschiedlichen oder aber auch auf dem gleichen Rechner haben. Die Systeme stellen je eigene Prozesse dar, die keinen gemeinsamen
Speicher haben und so autonom agieren können. Besonders wichtig ist die Koordinierung der Systeme und die Kommunikation zwischen ihnen. 
\\\\
In den meisten Quellen wird unter dem Begriff verteiltes System ein verteiltes Anwendungsystem verstanden. Dies ist ein Softwaresystem, das das Prinzip der verteilten Systeme nutzt um ein 
Problem aus dem bereich der elektronischen Datenverarbeitung löst. Dies geschieht im alles im betrieblichen Rahmen. Allerdings gibt es auch weitere Systeme, wie zum Beispiel verteilte Informationssysteme.
Informationssysteme dienen der Regelung von betriebsinternen und -externen Prozessen, die dem Austausch von Informationen dienen. Weiterhin gibt es verteilte Betriebsysteme.
Die Funktionalität eines Betriebsystems wird auf mehrere Kerne verteilt. Zusätzlich gibt es viele weitere Arten von verteilten Systemen, wie Embedded Systems oder auch Energiesysteme. Eine konkrete Klassifizierung aller verteilten Systeme ist aktuell nicht eindeutig geklärt und es gibt hierfür unterschiedliche
Ansätze.
\\\\
Allgemein gibt es für verteilte Systeme jedoch einige Vorteile gegenüber herkömmlichen Systemen. Die Last auf die einzelnen Komponenten verteilt, sodass kürzere Lade-und antowrtzeiten erreicht werden.
Die Autonomität der Systeme bedingt eine höhere Verfügbarkeit. Fällt ein Syste aus, so sind die anderen hiervon nicht betroffen und können problemlos weiterarbeiten. Die Skalierbarkeit stellt
einen weiteren Vorteil dar

\section{Sicherheit}