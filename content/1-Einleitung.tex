%!TEX root = ../dokumentation.tex

\chapter{Einleitung}

In der Vergangenheit kam es bei großen Unternehmen wie Facebook, Microsoft, Visa- und
MasterCard zum Teil mehrmals zu einer Entwendung von Kundendaten. Durch Attacken wie
Buffer Overflows, Viren oder andere Angriffsvektoren werden Maschinen und Benutzer auf
der ganzen Welt bedroht. Die Entwicklung des Internets legte ein besonderes Augenmerk auf
den Bereich der Netzwerksicherheit. Trotz vieler Bemühungen von IT-Sicherheitsexperten
und dem vorhanden sein leistungsfähiger Sicherheitsprotokolle und kryptografischen
Modulen kann ein vollständig sicheres System immer noch nicht gewährleistet werden.
Oftmals ist in dem Zusammenhang mit Daten Leaks nicht unbedingt die Kommunikation
zwischen Client und Server, sondern die eigentliche Software am Datenverlust maßgeblich
beteiligt. Die Gründe für das Schreiben unsicherer Software liegen oftmals an der mangelnden
Wahrnehmung von Fehlern seitens der Softwareentwickler oder der mangelnden
Verwendung von konkreten Mustern (Patterns) zur Lösung von Sicherheitsproblemen. In der
Software Entwicklung wird oftmals durch Frameworks bereits zur Entwicklungszeit die
Möglichkeit mitgeliefert bestimmte Sicherheitsmechanismen zu verwenden. Der Entwickler
hat die Aufgabe diese verstehen und richtig einsetzten zu können. Ein deutlicher Trend ist
momentan im Autonomisierungsbereich zu beobachten. Mit dem steigenden Einsatz von
Software in z.B. Haushaltsgeräten wird sich das Thema Sicherheit noch verschärfen. Kernziel
der Ausarbeitung ist unter anderem das Aufzeigen der Zusammenhänge zwischen
Sicherheitsaspekten in der Infrastruktur der Software und anderer Bereiche mit einem Fokus
auf die verteilten Systeme. Der Sachverhalt wird am Beispiel von heutzutage weit
verbreiteten elektronischen Zahlungsmitten aufgezeigt.

\section{Einführung in verteilte Systeme}

\section{Sicherheit}