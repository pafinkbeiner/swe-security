%!TEX root = ../dokumentation.tex

\chapter{Einleitung}

In der Vergangenheit kam es bei großen Unternehmen wie Facebook, Microsoft, Visa- und
MasterCard zum Teil mehrmals zu einer Entwendung von Kundendaten. Durch Attacken wie
Buffer Overflows, Viren oder andere Angriffsvektoren werden Maschinen und Benutzer auf
der ganzen Welt bedroht. Die Entwicklung des Internets legte ein besonderes Augenmerk auf
den Bereich der Netzwerksicherheit. Trotz vieler Bemühungen von IT-Sicherheitsexperten
und dem vorhanden sein leistungsfähiger Sicherheitsprotokolle und kryptografischen
Modulen kann ein vollständig sicheres System immer noch nicht gewährleistet werden.
Oftmals ist in dem Zusammenhang mit Daten Leaks nicht unbedingt die Kommunikation
zwischen Client und Server, sondern die eigentliche Software am Datenverlust maßgeblich
beteiligt. Die Gründe für das Schreiben unsicherer Software liegen oftmals an der mangelnden
Wahrnehmung von Fehlern seitens der Softwareentwickler oder der mangelnden
Verwendung von konkreten Mustern (Patterns) zur Lösung von Sicherheitsproblemen. In der
Software Entwicklung wird oftmals durch Frameworks bereits zur Entwicklungszeit die
Möglichkeit mitgeliefert bestimmte Sicherheitsmechanismen zu verwenden. Der Entwickler
hat die Aufgabe diese verstehen und richtig einsetzten zu können. Ein deutlicher Trend ist
momentan im Autonomisierungsbereich zu beobachten. Mit dem steigenden Einsatz von
Software in z.B. Haushaltsgeräten wird sich das Thema Sicherheit noch verschärfen. 
\newline
Im Rahmen der Projektarbeit wird die Frage beantwortet, wie
verteilte Systeme sicher gestaltet und implementiert werden können. 
Zunächst sollen die Begriffe Sicherheit und verteilte Systeme definiert werden, 
um einen ersten Einblick in das Thema zu bieten. 
Hinführen soll die Einführung auf die Beantwortung der Frage, 
was vor wem im System geschützt werden muss. Auf dieser Basis kann beschrieben werden, 
was es für Angriffe auf verteilte Systeme gibt. 
Welche Sicherheitslücken können von Angreifern ausgenutzt werden? 
Weiterhin sollen Anforderungen an verteilte Systeme abgeleitet und konkret definiert werden. 
Welche Sicherheitsdienste werden benötigt, um diese Anforderungen umzusetzen? 
Neben einer allgemeinen Definition dieser Dienste soll auch ein konkreter Vergleich mit der tatsächlichen Praxis erfolgen. 
Wie erfolgt die Sicherstellung eines sicheren Systems in der Praxis? 
Für den Vergleich kann ein Beispiel herangezogen werden. 
Um zu zeigen, dass das theoretisch erläuterte funktioniert, soll ein Prototyp entwickelt werden. 
Der Prototyp soll abstrakt einige Sicherheitsdienste implementieren und so das zuvor erklärte demonstrieren. 
Zur Umsetzung der Prototypen sollen geeignete Technologien und Architekturen gesucht und evaluiert werden, 
sodass der beste und einfachste Ansatz ausgewählt wird.


\section{Einführung in verteilte Systeme}

\section{Sicherheit}