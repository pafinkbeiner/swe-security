%!TEX root = ../dokumentation.tex

\chapter{Einleitung}\label{Einleitung}

In der Vergangenheit kam es bei großen Unternehmen wie Facebook, Microsoft, Visa- und
MasterCard zum Teil mehrmals zu einer Entwendung von Kundendaten. Durch Attacken wie
Buffer Overflows, Viren oder andere Angriffsvektoren werden Maschinen und Benutzer auf
der ganzen Welt bedroht. Die Entwicklung des Internets legte ein besonderes Augenmerk auf
den Bereich der Netzwerksicherheit. Trotz vieler Bemühungen von IT-Sicherheitsexperten
und dem vorhanden sein leistungsfähiger Sicherheitsprotokolle und kryptografischen
Modulen kann ein vollständig sicheres System immer noch nicht gewährleistet werden.
Oftmals ist in dem Zusammenhang mit Daten Leaks nicht unbedingt die Kommunikation
zwischen Client und Server, sondern die eigentliche Software am Datenverlust maßgeblich
beteiligt. Die Gründe für das Schreiben unsicherer Software liegen oftmals an der mangelnden
Wahrnehmung von Fehlern seitens der Softwareentwickler oder der mangelnden
Verwendung von konkreten Mustern (Patterns) zur Lösung von Sicherheitsproblemen. In der
Software Entwicklung wird oftmals durch Frameworks bereits zur Entwicklungszeit die
Möglichkeit mitgeliefert bestimmte Sicherheitsmechanismen zu verwenden. Der Entwickler
hat die Aufgabe diese verstehen und richtig einsetzten zu können. Ein deutlicher Trend ist
momentan im Autonomisierungsbereich zu beobachten. Mit dem steigenden Einsatz von
Software in z.B. Haushaltsgeräten wird sich das Thema Sicherheit noch verschärfen. 
\newline
Im Rahmen der Projektarbeit wird die Frage beantwortet, wie
verteilte Systeme sicher gestaltet und implementiert werden können. 
Zunächst sollen die Begriffe Sicherheit und verteilte Systeme definiert werden, 
um einen ersten Einblick in das Thema zu bieten. 
Hinführen soll die Einführung auf die Beantwortung der Frage, 
was vor wem im System geschützt werden muss. Auf dieser Basis kann beschrieben werden, 
was es für Angriffe auf verteilte Systeme gibt. 
Welche Sicherheitslücken können von Angreifern ausgenutzt werden? 
Weiterhin sollen Anforderungen an verteilte Systeme abgeleitet und konkret definiert werden. 
Welche Sicherheitsdienste werden benötigt, um diese Anforderungen umzusetzen? 
Neben einer allgemeinen Definition dieser Dienste soll auch ein konkreter Vergleich mit der tatsächlichen Praxis erfolgen. 
Wie erfolgt die Sicherstellung eines sicheren Systems in der Praxis? 
Für den Vergleich kann ein Beispiel herangezogen werden. 
Um zu zeigen, dass das theoretisch erläuterte funktioniert, soll ein Prototyp entwickelt werden. 
Der Prototyp soll abstrakt einige Sicherheitsdienste implementieren und so das zuvor erklärte demonstrieren. 
Zur Umsetzung der Prototypen sollen geeignete Technologien und Architekturen gesucht und evaluiert werden, 
sodass der beste und einfachste Ansatz ausgewählt wird.


\section{Einführung in verteilte Systeme}\label{EinfuerungInVerteilteSysteme}

Für den Begriff \glqq verteilte Systeme \grqq{} liegt keine eindeutige Definition vor. Verschiedene Autoren definieren den Begriff der verteilten Systeme leicht unterschiedlich.
Nach A. Tanenbaum aus dem Jahre 2003 ist ein verteiltes System eine Menge voneinander unabhängiger Computer, die dem Benutzer wie ein einzelnes kohärentes System
erscheinen \cite{Mandl.2009}. Jeder Baustein kann seine Instanz auf unterschiedlichen oder aber auch auf dem gleichen Rechner haben. Die Systeme stellen je eigene Prozesse dar, die keinen gemeinsamen
Speicher haben und so autonom agieren können. Besonders wichtig ist die Koordinierung der Systeme und die Kommunikation zwischen ihnen. 
\\\\
In den meisten Quellen wird unter dem Begriff verteiltes System ein verteiltes Anwendungsystem verstanden. Dies ist ein Softwaresystem, das das Prinzip der verteilten Systeme nutzt um ein 
Problem aus dem bereich der elektronischen Datenverarbeitung löst. Dies geschieht im alles im betrieblichen Rahmen. Allerdings gibt es auch weitere Systeme, wie zum Beispiel verteilte Informationssysteme.
Informationssysteme dienen der Regelung von betriebsinternen und -externen Prozessen, die dem Austausch von Informationen dienen. Weiterhin gibt es verteilte Betriebsysteme.
Die Funktionalität eines Betriebsystems wird auf mehrere Kerne verteilt. Zusätzlich gibt es viele weitere Arten von verteilten Systemen, wie Embedded Systems oder auch Energiesysteme. Eine konkrete Klassifizierung aller verteilten Systeme ist aktuell nicht eindeutig geklärt und es gibt hierfür unterschiedliche
Ansätze.
\\\\
Allgemein gibt es für verteilte Systeme jedoch einige Vorteile gegenüber herkömmlichen Systemen. Die Last auf die einzelnen Komponenten verteilt, sodass kürzere Lade-und antowrtzeiten erreicht werden.
Die Autonomität der Systeme bedingt eine höhere Verfügbarkeit. Fällt ein Syste aus, so sind die anderen hiervon nicht betroffen und können problemlos weiterarbeiten. Die Skalierbarkeit stellt
einen weiteren Vorteil dar

\section{Sicherheit}\label{Sicherheit}
In diesem Abschnitt wird genauer auf den Sachverhalt eingegangen welche Gefahren \ac*{IT}-Systemen drohen.
Eine Konkretisierte Betrachtung der Angriffsvektoren wird in \autoref{Sicherheitsluecken} durchgeführt.
IT-Sicherheit ist ein Teil der Informationssicherheit und befasst sich mit der Planung, Maßnahmen und Kontrollen, die dem Schutz der IT dienen.
Sie reicht dabei vom Schutz einzelner Dateien bis hin zur Absicherung von Rechenzentren und Cloud-Diensten. 
Der Bereich von IT-Sicherheit ist grob in vier Teilbereiche einzugliedern: 

\begin{description}
    \item 1. Schutz von Informationen und IT-Systemen
    \item 2. Schutz von Vernetzungen
    \item 3. Schutz des Benutzers
    \item 4. Verhinderung von Schwachstellen
\end{description}
TODO: QUELLE

\paragraph{Schutz von Informationen und IT-Systemen}

Das auch als "Endpoint Security" bezeichnete Grundkonzept der IT-Sicherheit befasst sich mit dem durchführen 
Organisatorischer Maßnahmen die den unbefugten Zugriff auf Geräte verhindern soll. 
Die genaue Art der Geräte spielt dabei keine Rolle es kann sich um Notebooks, Tablets, PCs oder andere Geräte handeln. 
Geschützt werden die Endgeräte vor verschiedenen Arten von Schadsoftware oder vor unbefugten Systemzugriffen. 
Besonders durch trends in der Unternehmenskultur wie bspw. "Bring your own device", gewinnt der Schutz der Firmeneigenen 
IT-Systeme immer mehr Bedeutung. 
Einige Maßnahmen haben sich im Bereich der Endgerätsicherheit heraus gezeichnet: 

\begin{itemize}
    \item Malware-Schutz
    \item Anwendungsisolation
    \item URL-Filter
    \item Client-Firewalls
\end{itemize}
TODO: QUELLE

Durch den umfassende Schutzmaßnahmen in diesem Bereich der IT-Sicherheit kann ein Großteil von Sicherheitsrisiken bereits 
verhindert werden. Ebenfalls ist durch den Schutz der Endgeräte durch Sicherheitsmechanismen wie Firewalls zusätzlich zu den 
Anwendungen auch das Betriebssystem des Gerätes geschützt. 

\paragraph{Schutz von Vernetzungen}



\paragraph{Schutz des Benutzers}

\paragraph{Verhinderung von Schwachstellen}