%!TEX root = ../dokumentation.tex

\chapter{Einleitung}\label{Einleitung}

In der Vergangenheit kam es bei großen Unternehmen wie Facebook, Microsoft, Visa- und
MasterCard zum Teil mehrmals zu einer Entwendung von Kundendaten. Durch Attacken wie
Buffer Overflows, Viren oder andere Angriffsvektoren werden Maschinen und Benutzer auf
der ganzen Welt bedroht. Die Entwicklung des Internets legte ein besonderes Augenmerk auf
den Bereich der Netzwerksicherheit. Trotz vieler Bemühungen von IT-Sicherheitsexperten
und dem vorhanden sein leistungsfähiger Sicherheitsprotokolle und kryptografischen
Modulen kann ein vollständig sicheres System immer noch nicht gewährleistet werden.
Oftmals ist in dem Zusammenhang mit Daten Leaks nicht unbedingt die Kommunikation
zwischen Client und Server, sondern die eigentliche Software am Datenverlust maßgeblich
beteiligt. Die Gründe für das Schreiben unsicherer Software liegen oftmals an der mangelnden
Wahrnehmung von Fehlern seitens der Softwareentwickler oder der mangelnden
Verwendung von konkreten Mustern (Patterns) zur Lösung von Sicherheitsproblemen. In der
Software Entwicklung wird oftmals durch Frameworks bereits zur Entwicklungszeit die
Möglichkeit mitgeliefert bestimmte Sicherheitsmechanismen zu verwenden. Der Entwickler
hat die Aufgabe diese verstehen und richtig einsetzten zu können. Ein deutlicher Trend ist
momentan im Autonomisierungsbereich zu beobachten. Mit dem steigenden Einsatz von
Software in z.B. Haushaltsgeräten wird sich das Thema Sicherheit noch verschärfen. 
\newline
Im Rahmen der Projektarbeit wird die Frage beantwortet, wie
verteilte Systeme sicher gestaltet und implementiert werden können. 
Zunächst sollen die Begriffe Sicherheit und verteilte Systeme definiert werden, 
um einen ersten Einblick in das Thema zu bieten. 
Hinführen soll die Einführung auf die Beantwortung der Frage, 
was vor wem im System geschützt werden muss. Auf dieser Basis kann beschrieben werden, 
was es für Angriffe auf verteilte Systeme gibt. 
Welche Sicherheitslücken können von Angreifern ausgenutzt werden? 
Weiterhin sollen Anforderungen an verteilte Systeme abgeleitet und konkret definiert werden. 
Welche Sicherheitsdienste werden benötigt, um diese Anforderungen umzusetzen? 
Neben einer allgemeinen Definition dieser Dienste soll auch ein konkreter Vergleich mit der tatsächlichen Praxis erfolgen. 
Wie erfolgt die Sicherstellung eines sicheren Systems in der Praxis? 
Für den Vergleich kann ein Beispiel herangezogen werden. 
Um zu zeigen, dass das theoretisch erläuterte funktioniert, soll ein Prototyp entwickelt werden. 
Der Prototyp soll abstrakt einige Sicherheitsdienste implementieren und so das zuvor erklärte demonstrieren. 
Zur Umsetzung der Prototypen sollen geeignete Technologien und Architekturen gesucht und evaluiert werden, 
sodass der beste und einfachste Ansatz ausgewählt wird.


\section{Einführung in verteilte Systeme}\label{EinfuerungInVerteilteSysteme}

Für den Begriff \glqq verteilte Systeme \grqq{} liegt keine eindeutige Definition vor. Verschiedene Autoren definieren den Begriff der verteilten Systeme leicht unterschiedlich.
Nach A. Tanenbaum aus dem Jahre 2003 ist ein verteiltes System eine Menge voneinander unabhängiger Computer, die dem Benutzer wie ein einzelnes kohärentes System
erscheinen \cite{Mandl.2009}. Jeder Baustein kann seine Instanz auf unterschiedlichen oder aber auch auf dem gleichen Rechner haben. Die Systeme stellen je eigene Prozesse dar,
 die keinen gemeinsamen
Speicher haben und so autonom agieren können. Besonders wichtig ist die Koordinierung der Systeme und die Kommunikation zwischen ihnen.
\\\\
In den meisten Quellen wird unter dem Begriff verteiltes System ein verteiltes Anwendungsystem verstanden. Dies ist ein Softwaresystem, das das Prinzip der verteilten Systeme 
nutzt um ein 
Problem aus dem bereich der elektronischen Datenverarbeitung löst. Es gibt jedoch einige weitere Klassen von verteilten Systemen. Eine konkrete Einteilung ist noch nicht gegeben, sodass
mehrere Autoren unterschiedliche Klassifizierungsarten verfolgen. Die am häufigsten verwendete Klassifizierung ist die von A. Tanenbaum. Er unterteilt die verteilten Systeme
in drei Klassen ein. Zunächst nennt er die verteilten Computersysteme. Hierzu gehören Systeme wie Cloud- oder Grid-Computing. Dies sind jeweils Rechnersysteme die über LAN miteinander verbunden
sind und gemeinsam eine verteilte Anwendung unterstützen (vgl \cite{Mandl.2009}). Weiterhin nennt Tanenbaum verteilte Informationssysteme.
Informationssysteme dienen hauptsächlich der Regelung von betriebsinternen und -externen Prozessen, die dem Austausch von Informationen dienen \cite{Lackes.o.J.}. 
Hierunter werden auch die Anwendungsysteme eingeordnet. Als konkrete Beispiele lassen sich elektronische Bibliotheken oder Reisebuchungssysteme nennen.
Zuletzt gibt es verteilte pervasive Systeme. Hierunter werden kleine, batteriegetriebene oder auch mobile verteile Systeme eingeordnet (vgl \cite{Mandl.2009}).
\newline
Die verteilten Systeme schneiden neben der Informatik viele weiter Anwendungsdomänen an. Für das Finanz- und Vetriebswesen lassen sich eCommerce wie PayPal oder Ebay auflisten. 
Im Bereich Vertrieb und Logistik spielen heutzutage Navigationssysteme wie Google Maps eine große Rolle. Auch im Gesundheitswesen sind verteilte Systeme nicht mehr wegzudenken. Sie dienen unter
der Überwachung der Lebensfunktionalitäten von Patienten. Besonders in Zukunft werden verteilte Systeme für erneuerbare Energien, beispielsweise Windkraftwerke, außerordentlich wichtig sein. \cite{o.V.2011}
\\\\
Allgemein gibt es für verteilte Systeme jedoch einige Vorteile gegenüber herkömmlichen Systemen. Die Autonomie der Systeme bedingt eine verbesserte Ausfallsicherheit.
Fällt ein System aus, so sind die anderen hiervon nicht betroffen und können problemlos weiterarbeiten. 
Die Skalierbarkeit und Lastverteilung stellt einen weiteren Vorteil dar. Skalierbarkeit bedeutet dass die Last auf die einzelnen Komponenten verteilt, sodass kürzere Lade-und Antwortzeiten
erreicht werden. Größere rechenintensive Prozesse werden auf leistungsstarker Hardware ausgeführt, kleinere Prozesse auf schlechterer Hardware. Um das System zu ergänzen können problemlos weitere
Systeme oder Komponenten hinzugefügt werden. Die daraus erfolgende Flexibilität kommt der Anpassung an Anforderungen zu Gute. Das System kann nach Belieben geändert werden und auf Änderungen
in den Systemanforderungen schnell agieren. Zudem können Funktionalitäten des Systems auf mehrere Entwicklungsteams aufgeteilt werden. Jedes Team kann autonom voneinander arbeiten. Dies stellt
einen schnellen und effizienten Entwicklungsprozess sicher. Die entstehenden Teile hinter dem gesamten System kann dem Nutzer verborgen werden. Dies bezeichnet man als Verteilungstransparenz. So sieht der Nutzer
lediglich die Anwendung kennt aber keine genauen Hintergrundprozesse oder auftretende Fehler in einem Teilsystem. \cite{Mandl.2009}
\newline
Jedoch gibt es auch einige Nachteile die unter der Verwendung von verteilten Systemen auftreten können. Es können viele Abhängigkeiten zwischen Teilkomponenten entstehen, 
die sogar einem Single-Point-of-Failure entstehen lassen können. Unter einem SPOF versteht man eine Komponente, deren Ausfall den Ausfall des kompletten Systems mit sich zieht. Weiterhin kann
es zu Problemen in der Homogenität der Gesamtanwendung kommen. Dies entsteht durch Verwendung verschiedener Programmiersprachen oder unterschiedlichen Oberflächen Designs in den Teilsystemen. 
Weiterhin kann es zu Problemen in der Sicherheit kommen, die im nachfolgenden Abschnitt näher erläutert werden.

\section{Sicherheit}\label{Sicherheit}
In diesem Abschnitt wird genauer auf den Sachverhalt eingegangen welche Gefahren \ac*{IT}-Systemen drohen.
Eine Konkretisierte Betrachtung der Angriffsvektoren wird in \autoref{Sicherheitsluecken} durchgeführt.
IT-Sicherheit ist ein Teil der Informationssicherheit und befasst sich mit der Planung, Maßnahmen und Kontrollen, die dem Schutz der IT dienen.
Sie reicht dabei vom Schutz einzelner Dateien bis hin zur Absicherung von Rechenzentren und Cloud-Diensten. 
Der Bereich von IT-Sicherheit ist grob in folgenden vier Teilbereiche einzugliedern: 

Quelle: https://www.security-insider.de/it-security-umfasst-die-sicherheit-der-ganzen-it-a-578480/

\textbf{Schutz von Informationen und IT-Systemen}

Das auch als \glqq Endpoint Security \grqq{} bezeichnete Grundkonzept der IT-Sicherheit befasst sich mit dem durchführen 
Organisatorischer Maßnahmen die den unbefugten Zugriff auf Geräte verhindern soll. 
Die genaue Art der Geräte spielt dabei keine Rolle es kann sich um Notebooks, Tablets, PCs oder andere Geräte handeln. 
Geschützt werden die Endgeräte vor verschiedenen Arten von Schadsoftware oder vor unbefugten Systemzugriffen. 
Besonders durch trends in der Unternehmenskultur wie bspw. \glqq Bring your own device \grqq{}, gewinnt der Schutz der Firmeneigenen 
IT-Systeme immer mehr Bedeutung. 
Einige Maßnahmen haben sich im Bereich der Endgerätsicherheit heraus gezeichnet: 

\begin{itemize}
    \item Malware-Schutz
    \item Anwendungsisolation
    \item URL-Filter
    \item Client-Firewalls
\end{itemize}
\cite{Dipl.Ing.FHStefanLuberPeterSchmitz.2020}

Durch den umfassende Schutzmaßnahmen in diesem Bereich der IT-Sicherheit kann ein Großteil von Sicherheitsrisiken bereits 
verhindert werden. Ebenfalls ist durch den Schutz der Endgeräte durch Sicherheitsmechanismen wie Firewalls zusätzlich zu den 
Anwendungen auch das Betriebssystem des Gerätes geschützt. 

\textbf{Schutz von Vernetzungen}

Netzwerkinfrastrukturen erstrecken sich heutzutage meistens über mehrere Geräte und Anwendungen hinaus. Der Schutz dieser 
Netzwerke wird allgemein auch als Netzwerksicherheit bezeichnet. Besonders im Zusammenhang mit verteilten Systemen muss auf diesem Punkt besonders Wert gelegt werden.
Es gilt die Systeme mit Verbindung ins Internet von Cyber-Bedrohungen abzuschirmen. Hierbei besteht das Ziel technische und organisatorische 
Maßnahmen so durchzuführen das die Integrität und Verfügbarkeit von Daten innerhalb eines Netzwerks und somit auch eines verteilten Systems stets gewährleistet werden. 
Innerhalb eines Netzwerkes hat sich eine Vielzahl an Techniken bereits etabliert. 
Zentraler Bestandteil für eine Sichere Kommunikation eines Netzwerks stellt dabei die Firewall dar. Firewalls kontrollieren den Datenfluss zwischen den 
Netzwerken, insbesondere zwischen dem Firmennetzwerk und dem Internet. Eine genauere Erläuterung der Schutzmaßnahmen und möglichen Angriffsvektoren im Zusammenhang mit 
der Vernetzung von mehreren Systemen wird zum späteren Zeitpunkt noch einmal genauer erläutert. 

Quelle: https://www.security-insider.de/was-bedeutet-netzwerksicherheit-a-578391/

\textbf{Schutz des Benutzers}\label{Schutz_des_Benutzers}

Der Anwender selbst ist auch Bestandteil der IT-Sicherheit. Es muss von vorne herein festgelegt werden welcher Benutzer auf welches System zugreifen darf. 
Das Festlegen solcher Richtlinien wird als \glqq Identity- und Access Management \grqq{} bezeichnet und regelt die zentrale Verwaltung von Identitäten und Zugriffsrechten auf unterschiedlichen
Systemen und Applikationen. Für die Erteilung von Zugriffsrechten muss sich ein Benutzer Authentifizieren und Autorisieren. 
Die Authentifizierung ist der Prozess bei dem der Benutzer dem System mittels Benutzerdaten bestätigt, dass er derjenige ist, für den er sich ausgibt. Die Autorisierung 
wird anschließend durchgeführt um die Systeme und Ressourcen festzulegen auf die der Benutzer Zugriff erhält. Das Identity- und Access Management hat die 
Aufgabe für eine Vereinfachung und Automatisierung der Prozesse zu sorgen. 

\textbf{Verhinderung von Schwachstellen}

Die meisten Bedrohungen in modernen IT-Systemen bestehen durch das Vorhandensein von Schwachstellen. Eine Aufgabe und ein Teilbereich der IT-Sicherheit sollte also auch 
das aufspüren und schließen von Sicherheitslücken sein. Oft treten solche Sicherheitslücken in Software auf, wenn ein Benutzer mehr machen kann als er eigentlich darf. 
Bei der eingesetzten Software muss durch Netzwerkadministratoren und Anwendungsbetreuer immer darauf geachtet werden, dass die Software immer auf dem aktuellesten Stand ist. 
\newline
Die Teilbereiche der IT-Sicherheit geben einen Eindruck welche Komponenten zu schützen sind. 
Die Quelle von Angriffen ist aber meistens noch wichtiger als das eigentliche Schutzziel. Aus diesem Grund wird eine genauere Betrachtung dahingehend 
erläutert, vor wem ein Schutz überhaupt notwendig ist. 
\newline
Das erste Bild das bei dem Gedanken eines Angreifers im IT-Umfeld entsteht, ist das Bild des Hackers. Allerdings sind diese meist nur ein Teil der Gefahren die auf IT-Systeme drohen. 
Auf IT-Systeme wirken die Naturgesetzte. Komponenten mit mechanischen Komponenten wie bspw. Festplatten oder Laufwerke sind besonders anfällig für Verschleiß. 
Ebenso wie die Wirkung der Naturgesetze stellen auch Naturkatastrophen eine Gefahr für IT-Systeme dar. In Firmen wird heutzutage bereits bei der Planung von 
IT-Infrastrukturen jede mögliche Eventualität bedacht und im Vorfeld auf Redundanz geachtet. Tritt so bspw. ein Feuer aus gibt es einen Rohrbruch oder eine Überschwemmung sollte 
eine Unterbrechungsfreie Redundanz der IT-Systeme weiterhin gewährleistet werden. 
\newline
Im Gegensatz zu natürlichen Ursachen kann auch der Mensch durch Unfähigkeit oder Nachlässigkeit eine Gefahr der IT-Sicherheit darstellen. 
Umso wichtiger ist es die Richtlinien für den Zugang an ein IT-System so granular wie möglich zu distanzieren \autoref{Schutz_des_Benutzers}.
\newline 
Auch andere IT-Systeme können die Sicherheit von verbundener IT-Systeme gefährden. Angenommen eine Person mit schlechter Intention versucht einen Zugang zu möglichst vielen 
IT-Systemen eines Netzwerks zu erlangen. Wenn es ihm gelingt eine Schwachstelle eines Systems zu erlangen erhält dieser die Möglichkeit trotz Firewallrichtlinien, einen Zugang 
auf alle weiteren Systeme zu erlangen, zu denen entsprechende Richtlinien in der Firewall festgelegt wurden. Dieses Verhalten wird heutzutage meist durch die Schadsoftwareart des 
\glqq Computerwurms \grqq{} realisiert. Nachdem dieser auf einem IT-System ausgeführt wurde hat es das Bestreben sich selbst, bspw. über das Netzwerk, zu vervielfältigen.
